\section*{Ход работы:}
\begin{enumerate}
    \item В спектре ртутной лампы лабораторная установка позволяет наблюдать интерференционные кольца от зелёной линии, двух жёлтых и одной фиолетовой, в спектре натриевой лампы — жёлтый дублет. С помощью катетометра измерим вертикальные координаты 5–6 диаметров для каждой спектральной линии. Пройдя центр, последовательно зафиксируем вторые координаты тех же колец. Пронумеровав предварительно кольца для каждой линии i = 1, ..., 5, 6 (i = 1 для кольца минимального диаметра), запишем соответствующие одному кольцу координаты друг под другом. Для спектра ртутной лампы измеряются диаметры зелёной и двух жёлтых линий (всего 30–36 отсчётов вертикальной координаты). \\
    Положение центра --- $170.52\degree$. \\
    \begin{tabu} to 1\textwidth{|c|c|c|c|c|c|c|c|c|c|c|}
    \hline
    Кольцо & 1 Ж & 1 З & 2 Ж & 2 З & 3 Ж & 3 З & 4 Ж & 4 З & 5 Ж & 5 З \\
    \hline
    Начало, $\degree$ & 165.74 & 165.35 & 163.96 & 161.22 & 160.32 & 158.54 & 157.77 & 156.32 & 155.57 & 154.50 \\
    \hline
    Конец, $\degree$ & 165.35 & 164.32 & 163.42 & 160.75 & 159.95 & 158.14 & 157.33 & 155.95 & 155.33 & 154.39 \\
    \hline
    \end{tabu}
    \item Параметры установки: $F_0 = 50$ мм, $F_1 = 110$ мм.
    \item Рассчитаем номер центрального кольца:
    \[ m = \frac{2L \cos{\theta}}{\lambda} \approx 336\]
    \item Рассчитаем дисперсионную область:
    \[ \Delta\lambda = \frac{\lambda}{m} \approx \frac{\lambda^2}{2L} \approx 1.8 \cdot 10^{-9} \]
    \item Рассчитаем расстояние $L$ между зеркалами:
    $L = \lambda 4 f^2 \frac{1}{k} \approx 0.12$ мм 
    \item Оценим аппаратную разрешающую способность интерферометра:
    \[ R = \frac{\lambda}{\delta\lambda} = \frac{4f^2}{D\delta r} \approx 1.5 \cdot 10^5\] 
    \item Рассчитаем теоретичские значения добротности и числа интерферирующих лучей:
    \[ Q \approx \frac{2\pi L}{\lambda(1- k)} \approx 9.2 \cdot 10^3\]
    \[ R = mN \rightarrow N = R / m \approx 4.5 \cdot 10^3 \]
\end{enumerate}