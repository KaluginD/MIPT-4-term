\documentclass[12pt]{article}
\usepackage[utf8]{inputenc}
\usepackage[russian]{babel}
\usepackage{titlesec}
\usepackage{titling}
\usepackage{textcomp}
\usepackage{graphicx}
\usepackage{amssymb}
\usepackage[russian]{babel}
\usepackage[babel = true]{microtype}
\usepackage[left=10mm,right=10mm, top=15mm,bottom=15mm]{geometry}
\usepackage{amsmath}
\usepackage{mathtools}
\usepackage{array}
\usepackage{wrapfig}
\usepackage{multirow}
\usepackage{gensymb}
\usepackage{tabu}
\usepackage{graphicx}
\usepackage{subcaption}
\let\oldAA\AA
\renewcommand{\AA}{\text{\normalfont\oldAA}}

\makeatletter
\newcommand*{\rom}[1]{\expandafter\@slowromancap\romannumeral #1@}
\makeatother

\setlength{\droptitle}{-5em}

\title{4.7.1 Двойное лучепреломление}
\date{}

\titleformat{\section}[hang]
{\normalfont\bfseries}
{\thesection.}{0.5em}{}

\titleformat{\subsection}[hang]
{\normalfont\bfseries}
{\thesection.}{0.5em}{}

\begin{document}

\maketitle

\paragraph{Цель работы:}изучение зависимости показателя преломления необыкновенной волны от направления в двоякопреломляющем кристалле; определение главных показателей преломления в кристалле.

\paragraph{В работе используются:}гелий-неоновый лазер, вращающийся столик с неподвижным лимбом, призма из исландского шпата, поляроид.

\section*{Теоретическая часть:}
\begin{enumerate}
    \item Величины $n_o = \sqrt{\varepsilon_{\bot}}, n_e = \sqrt{\varepsilon_{\|}}$ называют главными показателями
преломления кристалла. \\
Выразим показатель преломления необыкновенной волны $n = \varepsilon$ через главные показатели преломления $n_o , n_e$ и угол между оптической осью и волновым вектором $\theta$:
\[ \frac{1}{n^2(\theta)} = \frac{\sin^2{\theta}}{n^2_e} + \frac{\cos^2{\theta}}{n^2_o} \]
Заметим, что при $\theta = \pi/2$ показатель преломления необыкновенной волны равен $n = n_e$, а при $\theta = 0$ он равен $n = n_o$. Для обыкновенной волны показатель преломления равен $n_o$ независимо от направления её распространения. \\
Если $n_o − n_e \ll n_o, n_e$ (для исландского шпата $n_o = 1,655, n_e = 1,485$
для $\lambda = 0,63$ мкм), формулу можно упростить:
\[ n(\theta) \approx n_e + (n_o - n_e) \cos^2{\theta}\]

\subsection*{Двойное лучепреломление в призме из исландского шпата.}
    \begin{figure}[h!]
    \centering
    \begin{subfigure}[b]{0.25\textwidth}
        \includegraphics[width=\textwidth]{013.png}
    \end{subfigure}
    \begin{subfigure}[b]{0.25\textwidth}
        \includegraphics[width=\textwidth]{017.png}
    \end{subfigure}
    \caption\centering{а) Исследуемая призма из исландского шпата. Штриховкой указано направление оптической оси кристалла. б) Ход лучей в поляризационной призме}
    \end{figure}
    Значение показателя преломления и угол, под которым преломи-
лась волна в призме, можно найти, измерив угол падения на входную
грань призмы $\varphi_1$ и угол $\varphi_2$ на выходе призмы (рис. 2). Запишем закон Снеллиуса для одной из волн применительно к первой и второй граням призмы:
\[\sin{\varphi_1} = n \sin{\beta_1} \]
\[\sin{\varphi_2} = n \sin{\beta_2} = n \sin{(A - \beta_1)} \]
    \begin{figure}[h!]
        \noindent\centering{
            \includegraphics[height = 4cm]{018.png}
        }
        \caption{Ход лучей в призме}
    \end{figure}
    При этом мы выразили угол падения на вторую грань призмы $\beta_2$ через угол преломления на первой грани призмы $\beta_1$ и угол при вершине призмы $A$. Как видно из рис. 2, эти углы связаны простым соотношением: $A = \beta_1 + \beta_2$.\\
    Учитывая, что угол преломления $\beta_1$ связан с углом $\theta$ между осью кристалла и волновой нормалью $N$ соотношением $\theta + \beta_1 = \pi / 2$, находим $n$ и $\theta$:
\[ n = \frac{1}{\sin{A}} \sqrt{\sin^2{\varphi_1} + \sin^2{\varphi_2} + 2\sin{\varphi_1}\sin{\varphi_2}\cos{A} } \]
\[ \cos{A} = \frac{\sin{\varphi_1}}{n} \]    
    Для обыкновенной волны $n$ не будет зависеть от угла $\theta$, а для необыкновенной волны зависимость $n$ от $\theta$ должна описываться первой формулой. \\
    Показатель преломления призмы из изотропного материала удобно
находить по углу наименьшего отклонения луча от первоначального направления. Угол отклонения луча призмой ($\psi$ на рис. 2) минимален для симметричного хода лучей, т.е. когда $\varphi_1 = \varphi_2$ . Тогда показатель преломления можно рассчитать по формуле
\[n = \frac{\sin{\frac{\psi_m + A}{2}}}{\sin{\frac{A}{2}}} \]
где $\psi_m$ — угол наименьшего отклонения.    
\end{enumerate}
%\newpage
\section*{Экспериментальная установка}
\begin{figure}[h!]
        \noindent\centering{
            \includegraphics[height = 4cm]{019.png}
        }
        \caption{Схема экспериментальной установки}
    \end{figure}
    Источником излучения служит Не—Nе-лазер
($\lambda = 0,63$ мкм). \\
Угол падения $\varphi_1$ определяется по положению луча, отражённого от передней (входной) грани призмы (рис. 3). Из рис. 2 можно получить
связь углов $\varphi_1, \varphi_2$:
\[\varphi_2 = A + \psi - \varphi_1 ,\]
а угол $\psi$ — отклонение преломлённого луча от первоначального направления — определяется по разности отсчётов на лимбе между точками, куда попадает луч в отсутствие призмы, и точкой, куда попадает преломлённый луч.
\section*{Ход работы:}
\begin{enumerate}
    \item В спектре ртутной лампы лабораторная установка позволяет наблюдать интерференционные кольца от зелёной линии, двух жёлтых и одной фиолетовой, в спектре натриевой лампы — жёлтый дублет. С помощью катетометра измерим вертикальные координаты 5–6 диаметров для каждой спектральной линии. Пройдя центр, последовательно зафиксируем вторые координаты тех же колец. Пронумеровав предварительно кольца для каждой линии i = 1, ..., 5, 6 (i = 1 для кольца минимального диаметра), запишем соответствующие одному кольцу координаты друг под другом. Для спектра ртутной лампы измеряются диаметры зелёной и двух жёлтых линий (всего 30–36 отсчётов вертикальной координаты). \\
    Положение центра --- $170.52\degree$. \\
    \begin{tabu} to 1\textwidth{|c|c|c|c|c|c|c|c|c|c|c|}
    \hline
    Кольцо & 1 Ж & 1 З & 2 Ж & 2 З & 3 Ж & 3 З & 4 Ж & 4 З & 5 Ж & 5 З \\
    \hline
    Начало, $\degree$ & 165.74 & 165.35 & 163.96 & 161.22 & 160.32 & 158.54 & 157.77 & 156.32 & 155.57 & 154.50 \\
    \hline
    Конец, $\degree$ & 165.35 & 164.32 & 163.42 & 160.75 & 159.95 & 158.14 & 157.33 & 155.95 & 155.33 & 154.39 \\
    \hline
    \end{tabu}
    \item Параметры установки: $F_0 = 50$ мм, $F_1 = 110$ мм.
    \item Рассчитаем номер центрального кольца:
    \[ m = \frac{2L \cos{\theta}}{\lambda} \approx 336\]
    \item Рассчитаем дисперсионную область:
    \[ \Delta\lambda = \frac{\lambda}{m} \approx \frac{\lambda^2}{2L} \approx 1.8 \cdot 10^{-9} \]
    \item Рассчитаем расстояние $L$ между зеркалами:
    $L = \lambda 4 f^2 \frac{1}{k} \approx 0.12$ мм 
    \item Оценим аппаратную разрешающую способность интерферометра:
    \[ R = \frac{\lambda}{\delta\lambda} = \frac{4f^2}{D\delta r} \approx 1.5 \cdot 10^5\] 
    \item Рассчитаем теоретичские значения добротности и числа интерферирующих лучей:
    \[ Q \approx \frac{2\pi L}{\lambda(1- k)} \approx 9.2 \cdot 10^3\]
    \[ R = mN \rightarrow N = R / m \approx 4.5 \cdot 10^3 \]
\end{enumerate}
%\section*{Обработка результатов:}
\begin{enumerate}
    \item 
    \begin{align*}
        &R = \frac{\lambda}{\delta\lambda} = b \frac{dn}{d\lambda} \\
        &D = \frac{m}{\sqrt{d^2 - m^2 \lambda^2}} \\
        &m \cdot N = R \\
        &N = d \cdot n, \\
    \end{align*}
    где $m$ --- максимумы, $d = 1/n, n = 100$ штр/мм, $d$ --- длинна решетки. 
    
    \item Для каждой длинны волны определим показатель преломления: \\
    \begin{tabu} to 1\textwidth{|c|c|c|c|c|}
    \hline
    \lambda, nm & 404,7 & 491,6 & 546,1 & 578 \\
    \hline
    \delta  & 63\degree32'52" & 65\degree21'52" & 64\degree46'54" & 63\degree56'54" \\
    \hline
    $n$     & 1,7179    & 1,7315    & 1,7272    & 1,7209 \\
    \hline
    \end{tabu}
    
    \item Соответсвие длинн волн и показателя преломления: \\
    \begin{tabu}{|c|c|c|}
        \hline
        \lambda, nm & $n$ & \\
        \hline
        589,3 & 1,67 & d \\
        \hline
        486,1 & 1,72 & f \\
        \hline
        656,3 & 1,7 & c \\
        \hline
    \end{tabu}
    
    \item Найдем среднюю диспрсию и коэффициент диспрсии: \\
    $D = 0,02$ \\
    $\nu = 33,5$
    
    \item Найдем разрешающую способность призмы: \\
    $R = \frac{\lambda}{\delta\lambda} = 1964,33$ \\
    $b_\text{рабочая} = 0,039$ \\
    $R = b \frac{dn}{d\lambda} = 3500$

    \item И наконец, построим график зависимости $n(\lambda)$:
    \begin{figure}[h!]
        \noindent\centering{
            \includegraphics[height = 10cm]{3.png}
        }
    \end{figure} \\
    
\end{enumerate}

\section*{Вывод:} Изучили зависимости показателя преломления необыкновенной волны от направления в двоякопреломляющем кристалле; научились определять главные показатели преломления в кристалле.

\end{document}