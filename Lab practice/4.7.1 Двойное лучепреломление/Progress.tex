\section*{Ход работы:}
\begin{enumerate}
    \item Юстируем установку.
    \item Определеяем угол $A$: $A = 38\degree$
    \item Определим разрешённое направление поляризатора. Глядя на яркое пятно от  лампы на столе, найдем минимум интенсивности --- $80\degree$ (разрешенное направление $\vec{E}$ вертикально).
    \item Определим обычновенные и необыкновенные волны. Получим изображение лучей(как показано на рисунке 3 --- не 2 точки, соответствующие 2 лучам, а 4 --- 2 более яркие, 2 менее яркие, в силу переотраженных лучей в призме).
    \item Вращая столикс призмой, снимаем показания и обраатываем их на компьютере.    
    
    \begin{tabu} to 1\textwidth{|c|c|c|c|c|c|c|c|}
    \hline
    Номер точки & 1 & 2 & 3 & 4 & 5 & 6 & 7 \\
    \hline
    $\varphi_1$ & 10 & 20 & 30 & 40 & 50 & 60 & 70 \\
    \hline
    $\psi_0$ & 33 & 28 & 27.5 & 28 & 29.5 & 33 & 38 \\
    \hline
    $\psi_e$ & 23 & 21 & 21 & 22.5 & 25 & 28 & 33 \\
    \hline
    $\varphi_{2o}$ & 61.2 & 46.2 & 35.7 & 26.2 & 17.7 & 11.2 & 6.2 \\
    \hline
    $\varphi_{2e}$ & 51.2 & 39.2 & 29.2 & 20.7 & 13.2 & 6.2 & 1.2 \\
    \hline
    $\theta_o$ & 83.95 & 77.95 & 72.43 & 67.22 & 62.4 & 58.52 & 55.53 \\
    \hline
    $\theta_e$ & 83.31 & 76.79 & 70.66 & 65.16 & 60.29 & 55.82 & 52.58 \\ 
    \hline
    $\cos^2{\theta_o} $ & 0.01 & 0.04 & 0.09 & 0.15 & 0.21 & 0.27 & 0.32 \\
    \hline
    $\cos^2{\theta_e} $ & 0.01 & 0.05 & 0.11 & 0.18 & 0.25 & 0.32 & 0.37 \\
    \hline
    $n_o $ & 1.647 & 1.638 & 1.656 & 1.66 & 1.653 & 1.659 & 1.66 \\
    \hline
    $n_e $ & 1.491 & 1.496 & 1.509 & 1.53 & 1.546 & 1.541 & 1.546 \\
    \hline
    \end{tabu}
    \item Построим графики зависимости $n(\cos^2{\theta})$.
\end{enumerate}