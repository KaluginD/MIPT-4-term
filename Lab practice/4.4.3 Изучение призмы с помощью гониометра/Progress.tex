\section*{Ход работы:}
\begin{enumerate}
    \item Настройка гониометра. \\
        \rom{1} Отсчет углов:
        $84 \degree 22' 15" \backsim 84 \degree 28' 15"$
    
    \item Изучение призмы. \\
        \rom{3} Измерение преломляющего угла:
        \begin{align*}
            &\alpha_1 = 358 \degree 49' 55" \\
            &\alpha_2 = 241 \degree 19' 55" \\
            &\alpha = \alpha_1 - \alpha_2 = 117 \degree 30' 00" \text{--- преломляющий угол призмы}
        \end{align*}
    
        \rom{4} Минимальный угол отклонения:
        \begin{align*}
            &\text{желтый} &&63\degree56'54"\   &&\lambda = 578nm \\
            &\text{зеленый} &&64\degree46'54"\  &&\lambda = 546,1nm \\
            &\text{голубой} &&65\degree21'52"\  &&\lambda = 491,6nm \\
            &\text{фиолетовый} &&63\degree32'52"\  &&\lambda = 404,7nm 
        \end{align*}
        
        \rom{5} Разрешающая способность:
        \begin{enumerate}
            \item желтый дуплет --- расстояние $\approx 1,5mm$
            \item длина основания призмы --- $b = 7 cm$
        \end{enumerate}
        
\end{enumerate}