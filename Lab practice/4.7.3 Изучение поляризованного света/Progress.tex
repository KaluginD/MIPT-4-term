\section*{Ход работы:}
\begin{enumerate}
    \item Определим разрешённые направления поляроидов.
    \begin{align*}
        \varphi_1 = 20 \\
        \varphi_2 = 160
    \end{align*}
    
    \item Определим угол Брюстера для эбонита \\
    Минимум интенсивности достигается при $\alpha = 58\degree$. Тогда имеем:
    $\tg{\alpha} = \frac{n_2}{n_1} = n$ \\
    $n = 1.6$
    
    \item Исследуем поляризацию света в преломленных и отраженных от стопы лучах. \\
    Установим стопу в таком положении, при котором свет падает на нее под углом Брюстера. \\
    Осветим стопу неполяризационным светом и рассмотрим через поляроиды. \\
    Видим, что в отраженном луче свет поляризован вертикально.
    
    \item Определим главных направлений стопы. \\
    Поставим пластину и поляроиды $P_1$ и $P_2$. Вращая пластины 1 и 2, находим минимумы интенсивности: \\
    $\alpha_1 = 112\degree,\ \alpha_2 = 110\degree$
    
    \item Выделение пластин $\lambda / 2$ и $\lambda / 4$. \\
    Установим разрешённое направление поляроида горизонтально, а главные направления исследуемой пластинки 
    — под углом $45\degree$ к горизонтали. С помощью второго поляроида определим, какую поляризацию имеет свет, прошедший пластинку. \\
    Пластинка 1 --- линейная, с переходом в квадрат --- $\lambda / 2$; \\
    Пластинка 2 --- круговая --- $\lambda / 4$.
    
    \item Определим направлений большей и меньшей скорости в пластинке $\lambda / 4$. \\
    Пластинка чувствительного оттенка не меняет поляризации зеленого света в условиях предыдущего опыта. \\
    При повороте рейтера оранжевый цвет переходиит в голубой. \\
    "Быстрая" ось соответствует голубому цвету.
    
    \item Интерференция поляризованных лучей. \\
    Пр вращении пластины наблюдаем, что при сохранении цвета меняется интенсивность. \\
    При вращении второго поляроида --- наоборот, меняется цвет.
    
\end{enumerate}